% Use only LaTeX2e, calling the article.cls class and 12-point type.

\documentclass[12pt]{article}

% Users of the {thebibliography} environment or BibTeX should use the
% scicite.sty package, downloadable from *Science* at
% http://www.sciencemag.org/authors/preparing-manuscripts-using-latex 
% This package should properly format in-text
% reference calls and reference-list numbers.

\usepackage{scicite}

\usepackage{times}

% The preamble here sets up a lot of new/revised commands and
% environments.  It's annoying, but please do *not* try to strip these
% out into a separate .sty file (which could lead to the loss of some
% information when we convert the file to other formats).  Instead, keep
% them in the preamble of your main LaTeX source file.


% The following parameters seem to provide a reasonable page setup.

\topmargin 0.0cm
\oddsidemargin 0.2cm
\textwidth 16cm 
\textheight 21cm
\footskip 1.0cm


%The next command sets up an environment for the abstract to your paper.

\newenvironment{sciabstract}{%
\begin{quote} \bf}
{\end{quote}}



% Include your paper's title here

\title{Materials and Methods: The drivers of phenological asynchrony vary globally}


% Place the author information here.  Please hand-code the contact
% information and notecalls; do *not* use \footnote commands.  Let the
% author contact information appear immediately below the author names
% as shown.  We would also prefer that you don't change the type-size
% settings shown here.

\author
{Drew E. Terasaki Hart,$^{1\ast}$ Ian J. Wang,$^{1}$\\
\\
\normalsize{$^{1}$Department of Environmental Science, Policy, and Management, University of California,}\\
\normalsize{Berkeley, CA 94720, USA}\\
\\
\normalsize{$^\ast$To whom correspondence should be addressed; E-mail:  drew.hart@berkeley.edu.}
}

% Include the date command, but leave its argument blank.

\date{}



%%%%%%%%%%%%%%%%% END OF PREAMBLE %%%%%%%%%%%%%%%%



\begin{document} 

% Double-space the manuscript.

\baselineskip24pt

% Make the title.

\maketitle 


\paragraph*{NIR_VP and SIF data}

Both the NIRVP and SIF datasets were preprocessed on Google Earth Engine
(GEE; (Gorelick et al., 2017). Our main dataset was MODIS-derived NIRVP …


To help validate our NIRVP maps, we ran an identical analysis using the
global, gridded SIF dataset produced by Yu et al. (Yu et al., 2019). This
is a spatially contiguous time series dataset, interpolated by ANN from
the spatially discontiguous SIF data estimated along OCO-2 orbital
swaths. This dataset was rigorously validated, internally and externally,
by the data authors, who found that it accurately captured the global
patterns present in the original OCO-2 retrievals and that it explained
upwards of XX% of the variation in contemporaneous Chlorophyll
Fluorescence Imaging Spectrometer (CFIS) aerial measurements. We
downloaded this data from its host site
(https://daac.ornl.gov/VEGETATION/guides/Global_High_Res_SIF_OCO2.html),
then uploaded it to GEE.

Given that the SIF dataset interpolates across orbital gaps, but that the
original authors did not explicitly validate the seasonal phenological
patterns of the interpolated data, we compared seasonality in the
interpolated, orbital-gap data to seasonality in another,
coarser-resolution SIF data. To do this, we extracted SIF time series
from the ANN-interpolated dataset at random points within OCO-2 orbital
gaps in three tropical realms (the Neotropics, tropical Africa, and
Indo-Pacific and Australia), then compared those values to
contemporaneous time series extracted from another high-resolution,
satellite-derived SIF dataset from the TROPOMI sensor (CITE). We used
tropical regions for this validation because those regions’ lack of a
pronounced thermal winter creates the possibility that seasonality there
exhibits spatially varying patterns that are not accurately recovered by
spatial interpolation from orbital swaths. If the interpolated dataset
adequately captures the true seasonal patterns of SIF within OCO-2
orbital gaps then its time series should explain the bulk of the
variation in the TROPOMI time series. Indeed, we find that the
correlation between these orbital-gap datasets is XX\% (BREAK DOWN BY
REALM) (SEE Fig SX; ADD SUPPLEMENTAL FIGURE).



\paragraph*{Estimation and validation of seasonal phenology}

To estimate the seasonality of stand-level photosynthesis at all pixels,
we ran a harmonic regression model in which the data values (SIF or
NIRVP) were predicted as a function of time. We used the following model:

$$val=\beta _0\ +\ \beta _tt\ +\ \beta _{\sin ,ann}\sin (t_{ann})\ +\ \beta _{\cos ,ann}\cos (t_{ann})\ +\ \beta _{\sin ,sem}\sin (t_{sem})\ +\ \beta _{\cos ,sem}\cos (t_{sem})\ +\ \epsilon$$

where  is the linear time component (expressed in fractional days), and 
and  are circular time components for the annual and semiannual
frequencies (expressed in radians). This model is algebraically
equivalent to detrending the data and running a Fourier transform that
includes both the annual and semiannual frequency components. We chose to
include the semiannual frequency because our preliminary, global analysis
revealed numerous regions with predominating bimodal seasonal patterns
(i.e. regions containing many pixels whose R2 values were higher in
semiannual-only harmonic regression models than in annual-only models;
see Fig. SXX), many of which were expected on the basis of well-known
regions with two rainy seasons. We found that the linear combination of
both components was complex enough to adequately reflect the seasonality
in unevenly bimodal sites (i.e. sites with two annual peaks or troughs of
differing heights or depths), without risk of overfitting by including
unjustifiable higher-frequency components.

PARAGRAPH OR TWO EXPLAINING THE FLUXNET VALIDATION PROCEDURE

\paragraph*{Data filtering}
We pushed both the SIF and NIRV datasets through various filtering steps,
including filtering of invalid land cover types, open water bodies,
data-deficient pixels, and pixels whose fitted seasonality was not
statistically significant. First, we used 500 meter, yearly, global MODIS
land cover data (MCD12Q1.006; CITE), classified according to the Annual
International Geosphere-Biosphere Programme’s (IGBP) classification
scheme (‘Land Cover Type 1’) to mask out all pixels within the cropland,
urban, crop-natural mosaic, permanent snow/ice, barren, and permanent
water body categories (i.e. all pixels with a value greater than 12).
This produced a conservatie dataset restricted to only fully ‘natural’
land cover types. RUN SAME ANALYSIS WITH LOOSER FILTERINING HERE, SINCE
IT’S ARGUABLE THAT SOME OF THESE CATEGORIES SHOULD BE COVERED BY THE
A.S.H. ANYWAY?

To filter out all cells with inadequate data availability we calculated
the percent missingness of the time series at each pixel, then dropped
all pixels with percent missingness greater than 75%. As expected, this
predominantly filters out pixels with high missingness because of
persistent cloud cover (e.g. cloud forest regions) or because of long
periods without adequate sunlight to stimulate photosynthesis (i.e.
high-latitude regions).

Finally, after running the global harmonic regression described in the
previous section, we used permutation tests to filter out pixels whose
fitted seasonality was insignificant. To do this, we ran XX harmonic
regressions, each time permuting the time-series stack of raster images
so as to scramble any true temporal (i.e. seasonal) pattern. The image of
R2 values was extracted from each permuted regression, compared to the R2
values from the true harmonic regression, and used to create a binary
image indicating which pixels had a greater R2 value in the permuted
regression than in the real regression. After running the full series of
permutations, we filtered out of our analysis any pixels for which the
fraction of permuted R2s greater than the true R2 exceeded our threshold
for significance (ɑ=0.01).

\paragraph*{Mapping of phenological diversity}
To map phenological diversity globally, we ran an empirical orthogonal functions (EOF)
analysis on a global data cube consisting of the daily, fitted characteristic
annual phenology time series of phenology at all pixels.
To avoid modeling two predominant but unimportant sources of variation in our data,
we used two pre-processing steps: 1.) given our interest in the relative
timing of phenology between sites, irrespective of the absolute value and amplitude of
the seasonal variation in our NIRv variable, we normalized each pixel's time series to
itself; 2.) given our desire to identify like pixel phenologies globally,
relative to each pixel's solar calendar (i.e., irrespective of the first-order
but uninteresting control of north-south hemispheric seasonal alternation),
we ROTATED SOUTHERN HEMISPHERE TIME SERIES BY ..., effectively transforming both
hemispheres to a common-basis solar calendar (henceforth, 'common-sun time series').
Then, following common practice, we standardized our pre-processed, input
array and ran EOF using cosine-latitude weights.
To visualize our results, we displayed each pixel in RGB colorspace by using
its [0-1]-normalized eigenvalues for the first three EOFs.
Finally, to allow interpretation of the RGB-depictiion of phenologies,
we chose regularly spaced coordinate trios within the colorspace,
randomly drew 100 global pixels within XXXXX distance of those trios,
and plotted the common-sun time series of those pixels as line plots annotating
a HSV-based colorwheel, to serve as a map legend (Figure 3).

\paragraph*{Calculation of phenological asynchrony}
Google Earth Engine is designed for carrying out a set of common spatial
analysis operations at scale, but it is not particularly well suited to
the calculating of a custom and complicated neighborhood metric such as
spatial asynchrony. For this reason, our Earth Engine implementation of
the asynchrony calculation could not be scaled past a neighborhood of
about 50 km (for the SIF data) or XX km (for the NIRVP data). Thus we
exported the results of our filtered harmonic regression as a series of
\textit{TFRecord} (TensorFlow Record) files, each containing a stack of images of
the regression coefficients for a number of rectangular geographic
regions. Earth Engine accepts a ‘kernel width’ argument for the export of
these files, which determines the extent of overlap between neighboring
regions. We set the kernel width to be double our target neighborhood
size (300 km; WHY/HOW CHOSE THIS?), thus outputting a series of files
whose new asynchrony images could be independently calculated. These new
asynchrony files were calculated in parallel on a supercomputer (UC
Berkeley’s Savio, using the savio3 partition, each node of which contains
a 2.1 GHz Skylake processor with 32 cores and 96 GB of RAM).

We calculated global images of two distinct asynchrony metrics, to be
able to compare and verify our results. These metrics were calculated,
pixel-wise, using the following algorithm:
\begin{enumerate}
    \item Use the regression coefficients to calculate the 365-day annual time series of fitted SIF or NIRV values for the focal pixel.
    \item Identify all pixels whose centerpoints are within the chosen neighborhood radius of the focal pixel (the ‘neighbor pixels’).
    \item For each neighbor pixel:
    \begin{enumerate}
        \item Calculate the fitted SIF or NIRV time series;
        \item Calculate and save the R2 of the regression between that time series and the focal pixel’s time series;
        \item Calculate and save the 365-dimensional Euclidean seasonal distance between that time series the focal pixel’s time series (after standardizing each time series);
        \item Calculate and save the geographic (geodesic) distance to the focal pixel;
    \end{enumerate}
    \item Derive two distinct asynchrony metrics, calculated as the absolute values of the slopes of the two master regressions of:
        \begin{enumerate}
            \item neighbor-wise R2s on neighbor-wise geographic distances (‘asynchronyR2’);
            \item neighbor-wise Euclidean seasonal distances on neighbor-wise geographic distances (‘asynchronyEuc’);
        \end{enumerate}
\end{enumerate}

To be able to assess the performance of our two asynchrony metrics, we
also saved the R2s from the two master regressions (‘R2R2’ and ‘R2Euc’)
and the sample size of the master regressions (‘n’), giving us a final
output TFRecord file containing two asynchrony metrics and three
evaluative metrics (‘asynchronyR2’, ‘asynchronyEuc’, ‘R2R2’, ‘R2Euc’, and
‘n’; the ‘asynchrony image’). The output collection of asynchrony image
TFRecord files was re-ingested into Earth Engine, which automatically
stitched them back into a global map. 

\subsection*{Calculation of physiographic covariates}
We used Earth Engine to produce a number of neighborhood-based
physiographic covariates, to
use as covariates in our global modeling of phenological asynchrony.
Cell-center latitude and longitude were generated as a pair of images using Earth
Engine’s Image.pixelLonLat() function. Altitude data were derived from
the global, hole-filled Shuttle Radar Topography Mission (SRTM) 90-meter
dataset, loaded from Earth Engine’s data catalogue (Jarvis et al., 2008),
then used to calculate the vector ruggedness metric (VRM; a measure of
topographic complexity) within 50-kilometer pixel neighborhoods.

We calculated distance to the nearest major river (DNMR) using the World
Wildlife Federation (WWF) HydroSHEDS Free Flowing Rivers Network, V1
dataset, a vectorized feature collection of global river courses and
pertinent attribute data that was loaded from the GEE data
catalog (Grill et al., 2019; Lehner et al., 2008). We filtered the
dataset to retain only major riverbeds, which we defined as all river
segments with river order (‘RIV_ORD’) ≤ 4, i.e. all segments with
long-term average discharge ≥ 100 cubic meters per second (Linke et al.,
2019) We then used Earth Engine’s FeatureCollection.distance() function
to calculate each pixel’s distance to the nearest segment in this river
dataset, using a maxError argument of 1000000 (1000 km), to ensure
accurate calculation for all pixels.

We used the Dinerstein et al. (CITE) global biome dataset to
calculate a global map of distance to the nearest ecotone.
We loaded this data from the GEE data catalog, rasterized the biome vector data,
then calculated for each pixel the distance to the nearest pixel within a different
biome type. For computational tractability, GEE can only calculate this distance
up to a maximum distance of XXXXXX km, which we adopted as our own maximum distance value
with which we replaced (using an R script) all missing values in the GEE output raster.
This creates a spatial variable that is artificially truncated and thus distributionally
unnatural, but given the potential of the information it includes within the 0-to-XXXX-km range
and the robustness to non-normal covariates of our downstream random-forest analysis,
we retained this variable in our overall covariate set.

We also produced a set of neighborhood-based bioclimatic covariates.
Most of these (precipitation, mean temperature, climatic water deficit...) were derived from the TerraClimate dataset (CITE),
but cloud cover was derived the MODIS XXXXXX cloud mask, following
the methods of CITE Wilson and Jetz (2016).
For all climatic variables, we loaded the data from the GEE data catalog,
then used the same GEE code used for NIR_{V} and SIF to calculate
and export global overlapping TFRecord tilesets of the 5 harmonic regression
coefficients. The terraclimate dataset had a particularly coarse temporal resolution (monthly), but a long temporal extent (43 years), giving us good confidence in our ability to fit a reasonable characteristic curves for the annual seasonality of each variable.
We then used the same Julia code used for NIR_{V} and SIF data to calculate
a global map of each variable's spatial seasonal asynchrony.

them as GeoTIFF rasters, for downstream analysis in R (Team, 2020).
Analysis of physiographic predictors of asynchrony
Random forest model (provide formula and explain, justify)
Whatever other models we decide to use (e.g. superlearner model, as
recommended by Maura?)


\subsection*{Analysis of asynchrony drivers}

To prepare our phenological asynchrony data and covariate data for analysis, we read all
rasters into R, warped and clipped all covariate rasters to the
spatial reference system of the phenological asychrony raster, stacked them together,
then masked out all pixels missing a value for any variable in the stack.
We then drew a stratified random sample totalling 1\% of our total raster pixels,
stratified by phenological asynchrony DECILES (to ensure adequate sampling of
the long right tail of our response variable's distribution), giving us
a XXXXXXX-pixel dataset (henceforth, our 'RF dataset') that we use to build and validate a pair of global
and geographically-weighted random forest models.

To build our global random forest model we used the XXXX package in R. We ran
the model 
        GIVE FORMULA
to 70\% of our RF dataset that we randomly sampled as a training dataset.
We used XXX-fold cross-validation (CV) to carry out conservative and rigorous
feature selection. GIVE RF PARAMETERS, ETC... We then used the withheld 30\% of our
RF dataset as test data to carry out model validation. We summarized the resulting model
and covariate set, using RMSE and percent purity to assess variable
importance and using partial dependence plots (PDPs; implemented in the XXXXX R package)
to probe the influence of the most important retained variables.

It is reasonable, on the basis of first principles, to expect that different
covariates could drive phenological asynchrony in different regions of the globe.
For example, spatial asynchrony in climatic water deficit and in minimum temperature
might predominate in Mediterranean climate regions, where lower-altitude sites
do not experience a deep winter freeze but reach extremes of water stress early
in the solar year, whereas high-altitude sites experience peak
water stress later in the year and have a deep freeze the prevents winter
photosynthesis; however, such influences might not apply in most tropical montane regions,
where the changing topographic modulation of surface atmospheric flow that changes
in predominant vector direction during the year could lead to spatial
asynchrony in precipitation or cloud cover that would control phenological asynchrony.
To allow for the possibility of spatially-varying covariate importance,
we used the reduced covariate set from our final global RF model to run a
geographic (i.e., geographically-weighted) RF model (CITE GEORGANOS), which
fits local RF models within a gridded global set of overlapping spatial kernels.
We used the same key parameters here as in our global model (XXX, XXX, XXX),
and used XXXX=XXXX and XXXX=XXXX to parameterize the spatial kernel to be used
for the local models. We built the model using the same 70\%-training, 30\%-test split
in our RF data, and we validated the model using the 30\% test data in the same way
as for the global model. To explore and interpret the results of this model,
we map variable importance for each of the XXXXX highest-importance variables
from our global model.


\subsection*{Regional analysis of seasonal vs climatic distance}
WRITE THIS UP, ONCE DECIDED


\paragraph*{Global patterns of phenology}








\bibliography{scibib}

\bibliographystyle{Science}



\end{document}






















} % end matmethods

\showmatmethods{} % Display the Materials and Methods section

\acknow{(NOTE: I NEED TO COMPLY WITH FLUXNET (AND AMERIFLUX, IF USED) CITATION REQUIREMENTS!) We thank D. Ackerly, L. Anderegg, A. Bishop, T. Dawson, J. Frederick, N. Graham, M. Kelly, M. Kling, N. Muchhala, P. Papper, A. Turner, E. Westeen, G. Wogan, and M. Yuan for feedback and guidance on various iterations of the simulations presented herein. We thank Berkeley Research Computing for providing access to the Savio computing cluster. Lastly, we thank M. Terasaki Hart, C. Nemec-Hart, G. Hart, J. Hart, and M. Tylka for supporting and encouraging a lifetime of curiosity about nature, and XXXXXXXXXXXXXXXXXX for the good grooves. D.E.T.H. was supported by an Emerging Challenges in Tropical Science Graduate Student Fellowship from the Organization for Tropical Studies, by a Tinker Field Resesarch Grant from the UC Berkeley Center for Latin American Studies, by a research equipment grant from IdeaWild, and by a Berkeley Fellowship (to D.E.T.H.). I.J.W. was supported by a National Science Foundation grant DEB1845682 (to I.J.W.).}

\showacknow{} % Display the acknowledgments section

% Bibliography
\bibliography{terasaki_hart_ch2}

\end{document}
