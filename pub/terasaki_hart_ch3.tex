\documentclass[9pt,twocolumn,twoside,lineno]{pnas-new}
% Use the lineno option to display guide line numbers if required.

% packages
\usepackage{courier}

\templatetype{pnasresearcharticle} % Choose template 
% {pnasresearcharticle} = Template for a two-column research article
% {pnasmathematics} %= Template for a one-column mathematics article
% {pnasinvited} %= Template for a PNAS invited submission

\title{Diversity and spatial asynchrony of phenological and climatic seasonality}

% Use letters for affiliations, numbers to show equal authorship (if applicable) and to indicate the correspond  ing author
\author[a,1]{Drew E. Terasaki Hart}
\author[a]{Ian J. Wang}

\affil[a]{Department of Environmental Science, Policy, and Management, University of California, Berkeley, CA 94720}

% Please give the surname of the lead author for the running footer
\leadauthor{Terasaki Hart} 

% Please include corresponding author, author contribution and author declaration information
\authorcontributions{D.E.T.H. conceived, designed and wrote the simulations and analysis, and wrote the manuscript. I.J.W. helped conceive and design the simulations and analysis, and cowrote the manuscript.}
\authordeclaration{The authors declare no competing interests.}
\correspondingauthor{\textsuperscript{1}To whom correspondence should be addressed. E-mail: drew.hart@berkeley.edu}

% At least three keywords are required at submission. Please provide three to five keywords, separated by the pipe symbol.
\keywords{phenology $|$ seasonality $|$ $|$ biogeography $|$ remote sensing $|$ NIR_{V}P} 

\begin{abstract}
Here, we use a robust, spatially consistent proxy of photosynthetic activity,
rigorously validated by a second, independent proxy and by global flux-tower derived
GPP measurements, to globally map diversity and spatial asynchrony in
the timing of stand-level phenology. We then explore the drivers of phenological
asynchrony using global maps of spatial averages, variation, and asynchrony
in bioclimatic variables and landscape phsyiognomy.
We unveil striking global patterns of phenological asynchrony
that broadly overlap with continental biodiversity hotspots in
, both in tropical montane regions and in global Mediterranean and semi-arid cliamtes. We report evidence that global variation in spatial phenological asynchrony is predominantly explained by spatial asynchrony in precipitation, ..., but that the explanatory factors vary from region to region in substantial and interpretable ways. Global mediterranean regions show strong phenological asynchrony, driven by ... that is rooted in the ... BRIEF DESCRIPTION OF CLIMATIC DYNAMICS. Conversely, some tropical montane regions, PARTICULARLY THOSE WITH UNBROKEN MOUNTAIN RANGES RUNNING ORTHOGONAL TO PREDOMINANT EASTERLY TRADEWINDS, show strong phenological asynchrony driven by ... These patterns suggest strong relevance to evolutionary biogeography...
\end{abstract}

% Please add a significance statement to explain the relevance of your work
\significancestatement{Authors must submit a 120-word maximum statement about the significance of their research paper written at a level understandable to an undergraduate educated scientist outside their field of speciality. The primary goal of the significance statement is to explain the relevance of the work in broad context to a broad readership. The significance statement appears in the paper itself and is required for all research papers.}

\dates{This manuscript was compiled on \today}
\doi{\url{www.pnas.org/cgi/doi/10.1073/pnas.XXXXXXXXXX}}

\begin{document}

\maketitle
\thispagestyle{firststyle}
\ifthenelse{\boolean{shortarticle}}{\ifthenelse{\boolean{singlecolumn}}{\abscontentformatted}{\abscontent}}{}

\section*{DRAFT CONTENT FOR DISSERTATION INTRO}
My third chapter departs from microevolutionary modeling to probe a
hypothesis that attempts to bridge micro- and macro- time scales. It is
theoretically plausible and conceptually satisfying that some spatially
structured landscape genetic processes could be stable over long-enough
timescales to have driven phylogeographic processes and generated
contemporary biogeographic patterns. Some of the most pronounced such
patterns (beta diversity patterns in tropical mountain ranges, the
latitudinal diversity gradient) have enticed hypotheses along these lines,
attempting to explain current distributions of biodiversity on the basis
of spatially variable speciation rates driven by predictable and
persistent environmentally structured processes. INTRODUCE ASH…

\dropcap{N}EED A FIRST SENTENCE HERE!
- \textit{TODO: REREAD/SKIM PAPERS I'VE COLLECTED FOR THIS CHAPTER, THEN CRAFT WELL-HONED 2-4 PARGRAPH INTRO!}
- The seasonal timing of photosynthetic activity, and how this varies both regionally and globally, have major implications for evolutionary biogeography. EXPLAIN AHS. OTHER IMPLCIATIONS?
However, little is known about this understudied field of research,
making it difficult for hypotheses to be articulated and studied within
specific species or ecosystems.
- Over the past 1-2 decades, dramatic improvements in space-based remote sensing of strong proxies of photosynthetic activity have given rise to a number of time series datasets at ecologically relevant spatial scales capable of resolving variation in GPP across ecotones (CITE SUN ET AL SCIENCE 2017).
- Brief explanation of SIF and NIRv methods, justifications, and caveats.

Here, we use a MODIS-derived NIRv (OR IS IT NIRVP?) time series from 2010-2020 (RIGHT YEARS?) to ...

- 1-2 paragraphs quickly summarizing methods

- Global phenological diversity
    - discuss map
    - mention broad similarity to global map of ppt seasonal modality, except in regions with seasonal deep freeze during which temporal distribution of ppt has no immediate effect of photosynthetic activity
    - mention use of SIF map (and validation of its seasonal patterns within orbital gaps using a second, independent SIF dataset)
    - mention broad agreement with SIF-based validation map, as well as understandable differences (cite supps)
- Global climatic seasonal diversity
    - discuss map and compare to global phenological diversity map
- validation of fitted phenology
    - discuss flux-tower validation results
    - general agreement
    - largest disagreement in regions of high asnychrony (some Med climates, some tropical locations)
        - patterns still strong similar, but timing and height of peaks wander
        - implications for our findings? real or not? think through this more...
        - ref. fig showing 3 CA sites (and multi sites in Australia maybe?)
- Global phenological asynchrony
    - discuss map
    - generally low asynchrony in temperate regions with prolonged winter freeze
    - high asynchrony in all major mediterran climate regions
    - high asynchrony in a number of tropical montane regions, esp. where extensive mountain ranges run approximately orthogonal to prevalent easterly tradewinds
- global climatic asynchrony
    - provide maps (at least in supps) and briefly discuss)
    - what else?
- global RF model results
    - initial vars, and those retained in final models
    - vars of top importance
- regional RF model results
    - discuss amount by which most important vars and their relationships to asynch change
    - highlight key takeaways (and cite results maps in supps)
    

- Warm and moist tropical locations are sometimes described as being 'aseasonal', and may largely seem so in comparison to the seasonal patterns of vegetation greenness in regions with seasonal deep freeze or severe drought, or from the perspective of human physiological experience, but many organisms in these locations still exhibit annual reproductive phenologies that are likely modulated by some seasonally varying environmental cue (CITE RADIATION STUFF, ETC), however slight in magnitude.
- In temperatre regions, seasonality is driven by large circannual fluctuation in solar zenith angle and incident radiation, This latitudinal variation in synoptic-scale  by the circannual latitudinal variation in . In Mediterranean climates, this synoptic-scale variation in 


\section*{Results}


\section*{Discussion}

- strong agreement with global patterns in precipitation seasonal modality, except in regions with deep winter freeze
- corroborate striking bimodality in photosynthetic activit in California, USA (cite Turner), and find similarly striking regional patterns in other Mediterranean climate regions (e.g., coastal Chile, South Africa, Australia)





\matmethods{

\sub    section*{NIR_VP and SIF data}

Both the NIRVP and SIF datasets were preprocessed on Google Earth Engine
(GEE; (Gorelick et al., 2017). Our main dataset was MODIS-derived NIRVP …


To help validate our NIRVP maps, we ran an identical analysis using the
global, gridded SIF dataset produced by Yu et al. (Yu et al., 2019). This
is a spatially contiguous time series dataset, interpolated by ANN from
the spatially discontiguous SIF data estimated along OCO-2 orbital
swaths. This dataset was rigorously validated, internally and externally,
by the data authors, who found that it accurately captured the global
patterns present in the original OCO-2 retrievals and that it explained
upwards of XX% of the variation in contemporaneous Chlorophyll
Fluorescence Imaging Spectrometer (CFIS) aerial measurements. We
downloaded this data from its host site
(https://daac.ornl.gov/VEGETATION/guides/Global_High_Res_SIF_OCO2.html),
then uploaded it to GEE.

Given that the SIF dataset interpolates across orbital gaps, but that the
original authors did not explicitly validate the seasonal phenological
patterns of the interpolated data, we compared seasonality in the
interpolated, orbital-gap data to seasonality in another,
coarser-resolution SIF data. To do this, we extracted SIF time series
from the ANN-interpolated dataset at random points within OCO-2 orbital
gaps in three tropical realms (the Neotropics, tropical Africa, and
Indo-Pacific and Australia), then compared those values to
contemporaneous time series extracted from another high-resolution,
satellite-derived SIF dataset from the TROPOMI sensor (CITE). We used
tropical regions for this validation because those regions’ lack of a
pronounced thermal winter creates the possibility that seasonality there
exhibits spatially varying patterns that are not accurately recovered by
spatial interpolation from orbital swaths. If the interpolated dataset
adequately captures the true seasonal patterns of SIF within OCO-2
orbital gaps then its time series should explain the bulk of the
variation in the TROPOMI time series. Indeed, we find that the
correlation between these orbital-gap datasets is XX\% (BREAK DOWN BY
REALM) (SEE Fig SX; ADD SUPPLEMENTAL FIGURE).

\subsection*{Estimation and validation of seasonal phenology}
To estimate the seasonality of stand-level photosynthesis at all pixels,
we ran a harmonic regression model in which the data values (SIF or
NIRVP) were predicted as a function of time. We used the following model:

$$val=\beta _0\ +\ \beta _tt\ +\ \beta _{\sin ,ann}\sin (t_{ann})\ +\ \beta _{\cos ,ann}\cos (t_{ann})\ +\ \beta _{\sin ,sem}\sin (t_{sem})\ +\ \beta _{\cos ,sem}\cos (t_{sem})\ +\ \epsilon$$

where  is the linear time component (expressed in fractional days), and 
and  are circular time components for the annual and semiannual
frequencies (expressed in radians). This model is algebraically
equivalent to detrending the data and running a Fourier transform that
includes both the annual and semiannual frequency components. We chose to
include the semiannual frequency because our preliminary, global analysis
revealed numerous regions with predominating bimodal seasonal patterns
(i.e. regions containing many pixels whose R2 values were higher in
semiannual-only harmonic regression models than in annual-only models;
see Fig. SXX), many of which were expected on the basis of well-known
regions with two rainy seasons. We found that the linear combination of
both components was complex enough to adequately reflect the seasonality
in unevenly bimodal sites (i.e. sites with two annual peaks or troughs of
differing heights or depths), without risk of overfitting by including
unjustifiable higher-frequency components.

PARAGRAPH OR TWO EXPLAINING THE FLUXNET VALIDATION PROCEDURE

\subsection*{Data filtering}
We pushed both the SIF and NIRV datasets through various filtering steps,
including filtering of invalid land cover types, open water bodies,
data-deficient pixels, and pixels whose fitted seasonality was not
statistically significant. First, we used 500 meter, yearly, global MODIS
land cover data (MCD12Q1.006; CITE), classified according to the Annual
International Geosphere-Biosphere Programme’s (IGBP) classification
scheme (‘Land Cover Type 1’) to mask out all pixels within the cropland,
urban, crop-natural mosaic, permanent snow/ice, barren, and permanent
water body categories (i.e. all pixels with a value greater than 12).
This produced a conservatie dataset restricted to only fully ‘natural’
land cover types. RUN SAME ANALYSIS WITH LOOSER FILTERINING HERE, SINCE
IT’S ARGUABLE THAT SOME OF THESE CATEGORIES SHOULD BE COVERED BY THE
A.S.H. ANYWAY?

To filter out all cells with inadequate data availability we calculated
the percent missingness of the time series at each pixel, then dropped
all pixels with percent missingness greater than 75%. As expected, this
predominantly filters out pixels with high missingness because of
persistent cloud cover (e.g. cloud forest regions) or because of long
periods without adequate sunlight to stimulate photosynthesis (i.e.
high-latitude regions).

Finally, after running the global harmonic regression described in the
previous section, we used permutation tests to filter out pixels whose
fitted seasonality was insignificant. To do this, we ran XX harmonic
regressions, each time permuting the time-series stack of raster images
so as to scramble any true temporal (i.e. seasonal) pattern. The image of
R2 values was extracted from each permuted regression, compared to the R2
values from the true harmonic regression, and used to create a binary
image indicating which pixels had a greater R2 value in the permuted
regression than in the real regression. After running the full series of
permutations, we filtered out of our analysis any pixels for which the
fraction of permuted R2s greater than the true R2 exceeded our threshold
for significance (ɑ=0.01).

\subsection*{Calculation of phenological asynchrony}
Google Earth Engine is designed for carrying out a set of common spatial
analysis operations at scale, but it is not particularly well suited to
the calculating of a custom and complicated neighborhood metric such as
spatial asynchrony. For this reason, our Earth Engine implementation of
the asynchrony calculation could not be scaled past a neighborhood of
about 50 km (for the SIF data) or XX km (for the NIRVP data). Thus we
exported the results of our filtered harmonic regression as a series of
\textit{TFRecord} (TensorFlow Record) files, each containing a stack of images of
the regression coefficients for a number of rectangular geographic
regions. Earth Engine accepts a ‘kernel width’ argument for the export of
these files, which determines the extent of overlap between neighboring
regions. We set the kernel width to be double our target neighborhood
size (300 km; WHY/HOW CHOSE THIS?), thus outputting a series of files
whose new asynchrony images could be independently calculated. These new
asynchrony files were calculated in parallel on a supercomputer (UC
Berkeley’s Savio, using the savio3 partition, each node of which contains
a 2.1 GHz Skylake processor with 32 cores and 96 GB of RAM).

We calculated global images of two distinct asynchrony metrics, to be
able to compare and verify our results. These metrics were calculated,
pixel-wise, using the following algorithm:
\begin{enumerate}
    \item Use the regression coefficients to calculate the 365-day annual time series of fitted SIF or NIRV values for the focal pixel.
    \item Identify all pixels whose centerpoints are within the chosen neighborhood radius of the focal pixel (the ‘neighbor pixels’).
    \item For each neighbor pixel:
    \begin{enumerate}
        \item Calculate the fitted SIF or NIRV time series;
        \item Calculate and save the R2 of the regression between that time series and the focal pixel’s time series;
        \item Calculate and save the 365-dimensional Euclidean seasonal distance between that time series the focal pixel’s time series (after standardizing each time series);
        \item Calculate and save the geographic (geodesic) distance to the focal pixel;
    \end{enumerate}
    \item Derive two distinct asynchrony metrics, calculated as the absolute values of the slopes of the two master regressions of:
        \begin{enumerate}
            \item neighbor-wise R2s on neighbor-wise geographic distances (‘asynchronyR2’);
            \item neighbor-wise Euclidean seasonal distances on neighbor-wise geographic distances (‘asynchronyEuc’);
        \end{enumerate}
\end{enumerate}

To be able to assess the performance of our two asynchrony metrics, we
also saved the R2s from the two master regressions (‘R2R2’ and ‘R2Euc’)
and the sample size of the master regressions (‘n’), giving us a final
output TFRecord file containing two asynchrony metrics and three
evaluative metrics (‘asynchronyR2’, ‘asynchronyEuc’, ‘R2R2’, ‘R2Euc’, and
‘n’; the ‘asynchrony image’). The output collection of asynchrony image
TFRecord files was re-ingested into Earth Engine, which automatically
stitched them back into a global map. 

\subsection*{Calculation of physiographic covariates}
We used Earth Engine to produce a number of physiographic variables, to
use as covariates in our global modeling of asynchrony. Cell-center
latitude and longitude were generated as a pair of images using Earth
Engine’s Image.pixelLonLat() function. Altitude data were derived from
the global, hole-filled Shuttle Radar Topography Mission (SRTM) 90-meter
dataset, loaded from Earth Engine’s data catalogue (Jarvis et al., 2008).


SARAH TO INCLUDE TERSE PARAGRAPH EXPLAINING CALCULATION OF VRM

Distance to the nearest major river (DNMR) was derived from the World
Wildlife Federation (WWF) HydroSHEDS Free Flowing Rivers Network, V1
dataset, a vectorized feature collection of global river courses and
pertinent attribute data that was loaded from Earth Engine’s data
catalogue (Grill et al., 2019; Lehner et al., 2008). We filtered the
dataset to retain only major riverbeds, which we defined as all river
segments with river order (‘RIV_ORD’) ≤ 4, i.e. all segments with
long-term average discharge ≥ 100 cubic meters per second (Linke et al.,
2019) We then used Earth Engine’s FeatureCollection.distance() function
to calculate each pixel’s distance to the nearest segment in this river
dataset, using a maxError argument of 1000000 (1000 km), to ensure
accurate calculation for all pixels.

We aggregated all resulting covariates to match the resolution and
registration of our independent variables (SIF and NIRV), then exported
them as GeoTIFF rasters, for downstream analysis in R (Team, 2020).
Analysis of physiographic predictors of asynchrony
Random forest model (provide formula and explain, justify)
Whatever other models we decide to use (e.g. superlearner model, as
recommended by Maura?)


\subsection*{Analysis}



} % end matmethods

\showmatmethods{} % Display the Materials and Methods section

\acknow{(NOTE: I NEED TO COMPLY WITH FLUXNET (AND AMERIFLUX, IF USED) CITATION REQUIREMENTS!) We thank D. Ackerly, L. Anderegg, A. Bishop, T. Dawson, J. Frederick, N. Graham, M. Kelly, M. Kling, N. Muchhala, P. Papper, A. Turner, E. Westeen, G. Wogan, and M. Yuan for feedback and guidance on various iterations of the simulations presented herein. We thank Berkeley Research Computing for providing access to the Savio computing cluster. Lastly, we thank M. Terasaki Hart, C. Nemec-Hart, G. Hart, J. Hart, and M. Tylka for supporting and encouraging a lifetime of curiosity about nature, and XXXXXXXXXXXXXXXXXX for the good grooves. D.E.T.H. was supported by an Emerging Challenges in Tropical Science Graduate Student Fellowship from the Organization for Tropical Studies, by a Tinker Field Resesarch Grant from the UC Berkeley Center for Latin American Studies, by a research equipment grant from IdeaWild, and by a Berkeley Fellowship (to D.E.T.H.). I.J.W. was supported by a National Science Foundation grant DEB1845682 (to I.J.W.).}

\showacknow{} % Display the acknowledgments section

% Bibliography
\bibliography{terasaki_hart_ch2}

\end{document}
